\documentclass{beamer}
\usepackage[utf8]{inputenc}
\usepackage{graphicx}

\newtheorem{definicion}{Definición}
\newtheorem{ejemplo}{Ejemplo}

%%%%%%%%%%%%%%%%%%%%%%%%%%%%%%%%%%%%%%%%%%%%%%%%%%%%%%%%%%%%%%%%%%%%%%%%%%%%%%%
\title[Presentación con Beamer del nímero $\pi$]{Número $\pi$ }
\author[Técnicas Experimentales]{Raquel Estefanía Espino Mantas}
\date[23-04-2014]{23 de abril de 2014}
%%%%%%%%%%%%%%%%%%%%%%%%%%%%%%%%%%%%%%%%%%%%%%%%%%%%%%%%%%%%%%%%%%%%%%%%%%%%%%%

%\usetheme{Madrid}
%\usetheme{Antibes}
%\usetheme{tree}
%\usetheme{classic}

%%%%%%%%%%%%%%%%%%%%%%%%%%%%%%%%%%%%%%%%%%%%%%%%%%%%%%%%%%%%%%%%%%%%%%%%%%%%%%%
\begin{document}
  
%++++++++++++++++++++++++++++++++++++++++++++++++++++++++++++++++++++++++++++++
\begin{frame}

  \includegraphics[width=0.15\textwidth]{img/ullesc}
  \hspace*{7.0cm}
  \includegraphics[width=0.16\textwidth]{img/fmatesc}
  \titlepage

  \begin{small}
    \begin{center}
     Facultad de Matemáticas \\
     Universidad de La Laguna
    \end{center}
  \end{small}

\end{frame}
%++++++++++++++++++++++++++++++++++++++++++++++++++++++++++++++++++++++++++++++

%++++++++++++++++++++++++++++++++++++++++++++++++++++++++++++++++++++++++++++++
\begin{frame}
  \frametitle{Índice}
  \tableofcontents[pausesections]
\end{frame}
%++++++++++++++++++++++++++++++++++++++++++++++++++++++++++++++++++++++++++++++


\section{Historia del número $\pi$}


%++++++++++++++++++++++++++++++++++++++++++++++++++++++++++++++++++++++++++++++
\begin{frame}

\frametitle{Historia del número $\pi$}

\begin{definicion}
A lo largo de la historia han sido muchas las formas utilizadas por el
ser humano para calcular aproximaciones cada vez más exactas del número $\pi$.
%
Es posible obtener una aproximación al valor de $\pi$ de forma geométrica.
De hecho, ya los griegos intentaron obtener sin éxito una solución exacta al 
problema del valor de $\pi$ mediante el empleo de regla y compás. 
El problema griego conocido como cuadratura del círculo o, lo que es lo mismo,
obtener un cuadrado de área igual al área de un círculo cualquiera, lleva implícito
el cálculo del valor exacto de $\pi$.
%
Una vez demostrado que era imposible la obtención de $\pi$ mediante el uso de regla 
y compás, se desarrollaron varios métodos aproximados. Dos de las soluciones son 
las debidas a Kochanski (usando regla y compás) y la de Mascheroni (empleando únicamente un compás).\par

\end{definicion}

\end{frame}
%++++++++++++++++++++++++++++++++++++++++++++++++++++++++++++++++++++++++++++++

\section{Método de Kochanski}

%++++++++++++++++++++++++++++++++++++++++++++++++++++++++++++++++++++++++++++++
\begin{frame}

\frametitle{Método de Kochanski}

\begin{definicion}
  \begin{itemize}
  \item
 Se dibuja una circunferencia de radio R. Se inscribe el triángulo equilátero OEG. Se
 traza una recta paralela al segmento EG que pase por A, prolongándola hasta que corte al 
 segmento OE, obteniendo D. Desde el punto D y sobre ese segmento se transporta 3 veces 
 el radio de la circunferencia y se obtiene el punto C. El segmento BC es aproximadamente 
 la mitad de la longitud de la circunferencia.
  \end{itemize}
\end{definicion}
\end{frame} 
\begin{frame}
\begin{definicion}
  \begin{itemize}
  \item
$$BC^2=AB^2+(3-DA)^2 \,\!
OF= \frac{\sqrt{3}}{2}
\frac{DA}{EF} = $$
$$=\frac{OA}{OF} \rightarrow \frac{DA}{1/2}=$$
$$=\frac{1}{\sqrt{3}/2} \rightarrow DA=\frac{\sqrt{3}}{3}$$
 \pause
 \item
Sustituyendo en la primera fórmula:
$$BC^2= 2^2+\left (3-\frac{\sqrt{3}}{3}\right )^2 \rightarrow$$
$$BC = \sqrt{40-6 \sqrt{3} \over 3}=3,141533...$$

  \end{itemize}
\end{definicion}

\end{frame}
%++++++++++++++++++++++++++++++++++++++++++++++++++++++++++++++++++++++++++++++

\section{Método de Mascheroni}

%++++++++++++++++++++++++++++++++++++++++++++++++++++++++++++++++++++++++++++++
\begin{frame}
\frametitle{Método de Mascheroni}

\begin{definicion}
  \begin{itemize}
  \item
Método desarrollado por Lorenzo Mascheroni: se dibuja una circunferencia de radio R y
se inscribe un hexágono regular. El punto D es la intersección de dos arcos de
circunferencia: BD con centro en A', y CD con centro en A. Obtenemos el punto E como 
intersección del arco DE, con centro en B, y la circunferencia. El segmento AE es 
un cuarto de la longitud de la circunferencia, aproximadamente.
  
  \end{itemize}
\end{definicion}
\end{frame}
\begin{frame}
\begin{definicion}
  \begin{itemize}
  \item
Demostración (suponiendo R = 1)
$$AD=AC=\sqrt{3} OD=\sqrt{3-1}=\sqrt{2}$$
$$BE=BD=\sqrt{(OD-MB)^2+MO^2} BE=BD=$$
$$=\sqrt{\left( \sqrt{2}-\frac{\sqrt{3}}{2} \right)^2+\frac{1}{4}}=\sqrt{3-\sqrt{6}}$$\par
  \pause
  \item
Por el teorema de Ptolomeo, en el cuadrilátero ABEB'
$$BB' \cdot AE=AB \cdot EB' + BE \cdot AB'
2 \cdot $$
$$AE= \sqrt{1+\sqrt{6}}+\sqrt{9-3 \cdot \sqrt{6}}=3,142399...$$
  
  \end{itemize}
\end{definicion}

\end{frame}

\section{Fórmula para calcular $\pi$}

\begin{frame}
\frametitle{Fórmula para calcular $\pi$}
\begin{definicion}
  \begin{itemize}
  \item
$\pi$ se puede calcular mediante integración:
%
$$\int_{0}^{1} \! \frac{4}{1+x^2}\, dx = 4(atan(1) -atan(0)) = \pi $$
%
Esta integral se puede aproximar numéricamente con una fórmula de cuadratura.
El valor aproximado de PI es:  3.14680051839 
El valor de PI con 35 decimales: 3.1415926535897931159979634685441852\par
  \end{itemize}
\end{definicion}
\end{frame}
%
%++++++++++++++++++++++++++++++++++++++++++++++++++++++++++++++++++++++++++++++

\section{Bibliografía}
%++++++++++++++++++++++++++++++++++++++++++++++++++++++++++++++++++++++++++++++
\begin{frame}
  \frametitle{Bibliografía}

  \begin{thebibliography}{10}

    \beamertemplatebookbibitems
    \bibitem[Plan Estudios, 2011]{plan}
    Documento de verificación del grado.
    (2011)

    \beamertemplatebookbibitems
    \bibitem[Guía Docente, 2013]{guia}
    Guía docente.
    (2013)
    {\small $http://eguia.ull.es/matematicas/query.php?codigo=299341201$}

    \beamertemplatebookbibitems
    \bibitem[URL: CTAN]{latex}
    CTAN. {\small $http://www.ctan.org/$}

  \end{thebibliography}
\end{frame}

%++++++++++++++++++++++++++++++++++++++++++++++++++++++++++++++++++++++++++++++
\end{document}